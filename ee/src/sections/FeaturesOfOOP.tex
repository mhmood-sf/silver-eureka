\section{Features of Object-Oriented Programming}

One feature common to many OOP languages (such as Java, Scala, TypeScript, Kotlin, etc.) is polymorphism. Polymorphism is commonly defined as types (or entities) whose operations are applicable to values of more than one type (\cite{cardelli}). Interestingly, this definition is broad enough to include most of the distinguishing features of typed OOP languages, such as generic programming, subtyping and operator/function overloading, as we will see later.

This definition, however, is too broad. Polymorphism can be further divided into different forms with more precise definitions, allowing us to better understand how they may act as barriers to type inference. In particular, three forms of polymorphism are discussed: parametric, subtype and ad-hoc polymorphism.
