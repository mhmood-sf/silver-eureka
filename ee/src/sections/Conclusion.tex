\section{Conclusion}

In conclusion, global type inference for statically typed OOP languages supporting parametric polymorphism is not possible, while global type inference for OOP languages supporting ad-hoc or subtype polymorphism is possible but complex. However, simple partial type inference for typed polymorphic OOP languages is still possible to a large extent, through techniques such as local type inference and its variants. In most cases, this partial type inference is sufficient to a large degree.

In fact, it is important to note that while type inference has many benefits in terms of programming language design, there are also cases where (global) type inference may not be desirable. Even in languages that do support global type inference, it is often discouraged in practical settings. This is due to the fact that type annotations can also serve the purpose of documenting code, and making it easier to understand. Moreover, type annotations also aid the compiler in analyzing the code and providing more useful and readable error messages (as opposed to error messages containing obscure type variables that are not very helpful to the programmer) (\cite{pierceturner}). In these cases, type annotations are encouraged, while type inference is relied upon only when code becomes verbose or unreadable. This situation is ideal for partial type inference techniques, making them quite useful and practical.
