\documentclass{article}

% Math Packages
\usepackage{amsmath}
\usepackage{amssymb}
\usepackage{amsfonts}

% Refs/Links
\usepackage{hyperref}
\hypersetup{
    colorlinks=true,
    linktoc=all,
    linkcolor=blue,
    citecolor=blue
}

% Font and encoding
\usepackage[utf8]{inputenc}
\usepackage[T1]{fontenc}
\usepackage{lmodern}
% \usepackage{newpxtext,newpxmath}

% Paragraph formatting
\setlength{\parindent}{0pt}
\setlength{\parskip}{1em}

% Document details
\title{\Large \textbf{IB CS Case Study Notes: Blockchain}}
\author{-}
\date{May 2021}

\begin{document}

\begin{titlepage}
    \maketitle
\end{titlepage}

\begin{center}
    \tableofcontents
\end{center}

\newpage

\section{Resources}
Links:
\begin{itemize}
    \item Watch this visual demo first: https://www.youtube.com/watch?v=\_160oMzblY8
    \item 3b1b video that goes into some more depth: https://www.youtube.com/watch?v=bBC-nXj3Ng4
    \item Another in-depth video: https://www.youtube.com/watch?v=Lx9zgZCMqXE
    \item https://www.khanacademy.org/computing/computer-science/cryptography/modern-crypt/v/the-fundamental-theorem-of-arithmetic-1
    \item https://www.khanacademy.org/economics-finance-domain/core-finance/money-and-banking/bitcoin/v/bitcoin-what-is-it
\end{itemize}

\section{Terminology}

\subsection{51\% Attack}
51\% attack refers to an attack on a blockchain - usually bitcoin's, for which
such an attack is still hypothetical - by a group of miners controlling more
than 50\% of the network's mining hashrate, or computing power. The attackers
would be able to prevent new transations from gaining confirmations, allowing
the to halt payments between some or all users. They would also be able to
reverse transactions that were completed while they were in control of the
network, meaning they could double-spend coins. They would almost certainly
not be able to create a new coins or alter old blocks, so a 51\% attack would
probably not destroy bitcoin or another blockchain-based currency outright,
even if it proved highly damaging.

\subsection{Block}
Blocks are files where data pertaining to the bitcoin network is permanently
recorded. A block records some or all of the most recent bitcoin transactions
that have not yet entered any prior blocks. Thus a block is like a page of a
ledge or record book. Each time a block is `completed', it gives way to the
next block in the blockchain. A block is thus a permanent store of records
which, once written, cannot be altered or removed.

\subsection{Blockchain}
A blockchain is a growing list of records, called blocks, that are linked using
cryptography. Each block contains a cryptographic hash of the previous block,
a timestamp, and transaction data (generally represented as a Merkle tree).

\subsection{Candidate block}
In a few words, a candidate block is a block that a mining node (miner) is
trying to mine in order to receive the block reward. So a candidate block may
described as a temporary block that will be either validated or discarded by
the network. Miners compete with each other to validate the next block and add
to the blockchain, but first, they have to create a candidate block to
participate in the mining competition.

Candidate blocks are created by miners by collecting and organizing multiple
unconfirmed transactions from the memory pool. The transactions are then hashed
to form a Merkle tree structure, which will eventually produce a Merkle root
(or root hash). The merkle root is a single hash that represents all previous
hashes of that tree, and therefore, all transactions that were included in
that particular block.

\subsection{Collision resistance}
The collision resistance property requires that two different input messages
should not hash to the same output. In other words, $h(x) \not= h(y)$. This
property is also known as strong collision resistance.

\subsection{Cryptocurrency}
Cryptocurrency is a type of digital currency that uses cryptography for
security and anti-counterfeiting measures. Public and private keys are often
used to transfer cryptocurrency between individuals.

As a counter-culture movement that is often connected to cyberpunks,
cryptocurrency is essentially a fiat currency. This means users must reach a
consensus about cryptocurrency's value and use it as an exchange medium.
However, because it is not tied to a particular country, its value is not
controlled by a central bank. With bitcoin, the leading functioning example of
cryptocurrency, value is determined by market supply and demand, meaning that
it behaves much like precious metals, like silver and gold.

\subsection{Cryptographic hash}
A cryptographic hash function is a mathematical function used in cryptography.
Typical hash functions take inputs of variable length to return outputs of
fixed length. A cryptographic hash function combines the message-passing
capabilities of hash functions with security properties.

\subsection{Determinism}
A hash procedure must be deterministic -- meaning that for a given input value
it must always generate the same hash value. In other words, it must be a
function of the data to be hashed, in the mathematical sense of the term. This
requirement excludes hash functions that depend on external variable
parameters, such as pseudo-random number generators or the time of day. It
also excludes functions that depend on the memory address of the object being
hashed in cases that the address may change during execution (as may happen
on systems that certain methods of garbage collection), although sometimes
rehashing of the item is possible.

The determinism is in the context of the reuse of the function. For example,
Python adds the feature that hash functions make use of a randomized seed that
is generated once when the Python process starts in addition to the input to
be hashed. The python hash is still a valid hash function when used within a
single run. But if the values are persisted (for example, written to disk)
they can no longer be treated as valid hash values, since in the next run the
random value might differ.

\subsection{Digital signature}
A digital signature is a mathematical technique used to validate the
authenticity and integrity of a message, software or digital document. The
digital equivalent of a handwritten signature or stamped seal, a digital
signature offers far more inherent security, and it is intended to solve the
problem of tampering and impersonation in digital communications. Digital
signatures can provide the added assurances of evidence of origin, identity
and status of an electronic document, transaction or message and can
acknowledge informed consent by the signer.

\subsection{Distributed consensus}
Distributed consensus refers to the elaborate, largely mathematically-based
game that the members of the bitcoin network use to keep in sync their tens of
thousands of individual duplicate copies of the entire set of transactions
that ever happened in the blockchain.

Some basic facts you need, in order to understand the general idea:
\begin{itemize}
    \item Nobody, or rather every member, is in charge of the bitcoin network.
    \item There are about 24,000 full nodes in the bitcoin network.
    \item Each `full node' keeps a complete copy of the entire database of
        transactions that have ever happened on the bitcoin network. That's
        called `the blockchain'.
    \item The data in the blockchain database is chunked up into groups of
        transcations. Each group is called a block. The data in each block
        includes a mathematical dependency on the data in the previous block,
        which links them together. That's the chain part.
    \item You can add a transaction to the bitcoin network by just asking any
        full node to add it. That node sends the transaction out to the rest
        of the network.
\end{itemize}

\subsection{Double-spend problem}
Double-spending is a problem in which the same digital currency can be spent
more than once. In other words, double-spending is an instance in which a
transaction uses the same input as another transaction that has already been
broadcast on the network. This is a flaw that is unique to digital currencies
because digital information is something that can be reproduced rather easily.
Digital currencies such as bitcoin, can be thought of as being a digital file.
If, for example, Bob has a file that has been saved locally to his computer,
there is nothing preventing Bob from simply copying the file as many times
as he wants, and sharing the file with multiple individuals. This same
principle can be applied to digital currencies. It is not ideal for the same
digital currency to be spendable more than once, because it can result in
inflation and a loss of trust in that currency, making it effectively
worthless.

\subsection{Entropy}
In cryptography, entropy is a measure of true randomness. An n-bit number
chosen uniformly at random with a perfect random number generator has n bits
of entropy, and entropy of other things can be computed in comparison to this
case. For example, 4 words chosen uniformly at random from a word list of 1024
words has 40 bits of entropy because you can represent each word by 10 bits
($2^10 = 1034$) and stick the 4 groups of 10 bits together to get a 40-bit
number chosen uniformly at random. When dealing with things chosen uniformly
at random you can also compute the entropy by calculating the base-2 logarithm
of the total possible outcomes, e.g. there are $6^{20}$ possible outcomes when
rolling a 6-sided die 20 times and then writing down the results one after
another (i.e. not summing or reordering them), so the result has $log_2 6^{20}
~= 51.7$ bits of entropy. If the result is in any way biased (like the sum of
dice rolls, which is very much not uniformly distributed), then you can still
calculate the entropy, but it's more difficult.

\subsection{Genesis block}
A genesis block is the first block of a block chain. Modern versions of bitcoin
number it as block 0, though very early versions counted it as block 1. The
genesis block is almost always hardcoded into the software of the applications
that utilize its block chain. It is a special case in that it does not
reference a previous block, and for bitcoin and almost all of its derivatives,
it produces an unspendable subsidy.

\subsection{Immutable transactions}
Immutability -- the ability for a blockchain ledger to remain a permanent,
indelible, and unalterable history of transactions -- is a definitive feature
that blockchain evangelists highlight as a key benefit. Immutability has the
potential to transform the auditing process into a quick, efficient,
cost-effective procedure, and bring more trust and integrity to the data
businesses use and share every day.

\subsection{Key pair generation}
Key generation is the process of generating keys for cryptography. They key is
used to encrypt and decrypt data whatever the data is being encrypted or
decrypted.

Modern cryptographic systems include symmetric-key algorithms (such as DES and
AES) and public-key algorithms (such as RSA). Symmetric-key algorithms use a
single shared key; keeping data secret requires keeping this key secret.
Public-key algorithms use a public key and a private key. The public key is
made available to anyone (often by means of a digital certificate). A sender
will encrypt data with the public key; only the holder of the private key can
decrypt this data.

\subsection{Ledger}
Ledge generally refers to the bull of quantities made in accounts.

Same in cryptoworld, it makes sense with the record of transactions being done
among bitcoin users.

Also it is a secured database which stores and holds the money of people in
the form of bitcoins.

\subsection{Merkle proof}
Merkle proofs are used to decide upon the following factors:
\begin{itemize}
    \item If the data belongs in the merkle tree.
    \item To concisely prove the validity of data being part of a dataset
        without storing the whole data set.
    \item To ensure the validity of a certain data set being inclusive in a
        larger data set without revealing either the complete data set or
        its subset.
\end{itemize}

\subsection{Merkle tree}
A merkle tree is a hash-based data structure that is a generalization of the
hash list. It is a tree structure in which each leaf node is a hash of a block
of data, and each non-leaf node is a hash of its children. Typically, merkle
trees have a branching factor of 2, meaning that each node has up to 2
children.

Merkle trees are used in distributed systems for efficient data verification.
They are efficient because they use hashes instead of full files. Hashes are
ways of encoding files that are much smaller than the actual file itself.
Currently, their main uses are in peer-to-peer networks such as Tor, bitcoin,
and git.

\subsection{Miner}
Miners can be defined as accountants who record every transaction to the
blockchain. The concept is simple, a proof of payment is important if you want
your payment to be valid. The miners are the ones who keep the records of your
payment. Hence they are record keepers who keep the system updated of new
payments and existing ones.

\subsection{Mining}
Bitcoing mining is the process of creating, or rather discovering, bitcoin
currency. Unlike real-world money that is printed when more is needed, bitcoin
cannot simple be willed into existence, but has to be mined through
mathematical processes. Bitcoin maintains a public ledger that contains past
transactions, and mining is the process of adding new transactions to this
ledger.

\subsection{Nonce}
A nonce (number only used once) is a number added to a hashed block that, when
rehashed, meets the difficulty level restrictions. The nonce is the number that
blockchain imners are solving for.

\subsection{Non-invertibility}
Non-invertibility is another feature that's often desirable, depending on the
intended usage of the algorithm. This says that it should be impossible, or at
least prohibitively difficult, to work out the input that led to any given
hash. Ideally, it should be easy to transform data into a hash, and practically
impossible to go the other way.

\subsection{Non-repudiation}
Non-repudiation is a method of guaranteeing message transmission between
parties via digital signatures and/or encryption. It is one of the five pillars
of information assurance (IA). The other four are availability, integrity,
confidentiality and authentication.

Non-repudiation is often used for digital contracts, signatures, and email
messages.

By using a data hash, proof of authentic identifying data and data origination
can be obtained. Along with digital signatures, public keys can be a problem
when it comes to non-repudiation if the message recipient has exposed, either
knowingly or unknowingly, their encrypted or secret key.

\subsection{One-way function}
A hash is designed to act as a one-way function -- you can put data into a
hashing algorithm and get a unique string, but if you come upon a new hash,
you cannot decipher the input data it represents. A unique piece of data will
always produce the same hash.

\subsection{Proof of work}
A proof of work is a piece of data which is difficult (costly, time-consuming)
to produce but easy for other to verify and which satisfies certain
requirements. Producing a proof of work can be a random process with low
probability so that a lot of trial and error is required on average before a
valid proof of work is generated. Bitcoin uses the hashcash proof of work
system.

\subsection{PuTTYgen}
PuTTYgen is a key generator. It generates pairs of public and private keys to
be used with WinSCP. PuTTYgen generates RSA, DSA, ECDSA, and Ed25519 keys.

\subsection{Self-referential data structure}
A self-referential class contains a reference member that refers to a class
object of the same class type.

Self-referential objects can be linked together to form useful data structures
such as lists, queues, stacks and trees.

\subsection{SHA256}
The Secure Hash Algorithm (SHA) is one of a number of cryptographic hash
functions. A cryptographic hash is like a signature for a text or a data file.
SHA-256 algorithm generates an almost-unique, fixed size 256-bit (32-byte)
hash. Hash is a one way function - it cannot be decrypted back. This makes it
suitable for password validation, challenge hash authentication, anti-tamper,
digital signatures, etc.

SHA-256 is one of the successor hash functions to SHA-1, and is one of the
strongest hash functions available.

\subsection{Takeover attack}
Account takeover is a form of identity theft where a fraudster illegally gets
access to a victim's bank or online e-commerce account using bots. A successful
account takeover attack leads to fraudulent transactions and unauthorized
shopping from the victim's compromised account.

\subsection{Transaction pool}
The transaction pool (or mempool, as it's usually called) is not a network-wide
pool. Each node maintains its own mempool. When a node receives your
transaction, it will validate and add it to its own mempool, and possible
broadcast further on.

\section{Other terms}

\subsection{Asymmetric key cryptography}
Asymmetric key cryptography is used when signing a transaction and verifying
the identity of the sender. Signing uses a private key which is kept secret,
in conjunction with a message (i.e. transaction / ledger entry) to produce a
unique signature for that message. Anyone can then use the sender's publicly
available verification key (public key) with a verification function. They
need to input a public key, the message and the signature, and it outputs
whether it was really the sender who signed it with absolute certainty.

\subsection{Avalanche effect}
A property of good hash functions. When a small change to a message should
change the hash value so extensively that the new hash value appears entirely
different from the old one.

\subsection{Salt}
A secret random number that is appended (added to the end) of an input to a
hash function. It serves as an additional safeguard against someone breaking
it (finding the original input data through trial and error). A salt is used
to make common passwords (e.g. “password”) generate different keys, but it
doesn't have to be unique, but is kept secret, unlike a nonce.

\subsection{Generation transaction/coinbase (reward)}
Always the first transaction in a block, and contains a predetermined reward
for the miner who found the proof of work for the block.

\subsection{Rainbow table}
It is a precomputed table of inputs for a hash function. It is usually used to
crack passwords that have been hashed without using a salt by searching for a
hash in it (which is quicker than computing hashes using brute force).


\section{Possible Questions}
Q. Explain how a cryptocurrency transaction is validated and how the blockchain
is involved.

Q. Explain how a program would traverse a [given] self-referential data
structure and display the data in each node (using pseudo-code or a flowchart).

Q. Explain how you would add a new node between two [given] nodes in a [given]
self-referential data structure (using pseudo-code or a flowchart).

Q. Explain how a self-referential data structure could be used to implement a
blockchain.

Q. Explain what a hashing algorithm is.

Q. The essential characteristics of good hashing algorithms are determinism,
noninvertibility and collision resistance. Explain the meaning of these terms.

Q. Explain what ``asymmetric key cryptography'' means in the context of a
blockchain.

Q. Digital signatures are used to validate MONS transactions before they are
added to the transaction pool. In this context, explain:
\begin{itemize}
    \item key generation
    \item creation of a signature
    \item verification of a signature
\end{itemize}

Q. Key generation software, such as PuTTYgen, often uses a physical source of
entropy to generate key pairs. Explain the use of entropy in this context.

\section{Sources}
The terminology definitions were taken from: https://quizlet.com/409294126/ib-computer-science-case-study-2020-a-local-economy-driven-by-blockchain-flash-cards

Other terms and possible questions were taken from documents made by strox\#4591

\end{document}

