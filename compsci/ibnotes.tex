\documentclass{article}

% TODO: Fix sub/subsubsections for T4 onwards

% Math Packages
\usepackage{amsmath}
\usepackage{amssymb}

% Refs/Links
\usepackage{hyperref}
\hypersetup{
    colorlinks=true,
    linktoc=all,
    linkcolor=blue,
    citecolor=blue
}

% Font and encoding
\usepackage[utf8]{inputenc}
\usepackage[T1]{fontenc}
\usepackage{lmodern}
% \usepackage{newpxtext,newpxmath}

% Paragraph formatting
\setlength{\parindent}{0pt}
\setlength{\parskip}{1em}

% Document details
\title{\Large \textbf{IB Computer Science Notes}}
\author{}
\date{May 2021}

\begin{document}

\begin{titlepage}
\maketitle
\end{titlepage}

\begin{center}
    \tableofcontents
\end{center}

\newpage
\section*{Note}
These notes were compiled using content from the textbook ``Core Computer
Science for the IB Diploma Program'' by Kostas Dimitriou \& Markos Hatzitaskos.
I've tried to follow the IB CS Guide as closely as possible, and sections and
subsections are divided according to the guide, as closely as possible (
subsubsections, however, do not follow the guide).

For topic 4, despite being the largest SL topic, the section has very little
actual content worth putting into notes. It's mostly just practicing
algorithms, pseudocode and so on, which is why I'd recommend putting more time
into practicing, either with past papers or other resources, rather than
using notes. Do remember to take a look at the IB pseudocode guides
as well. There are two documents, one regarding pseudocode notation
conventions, will be available to candidates during the examination,
and another regarding standard data structures that may appear on external
examinations. This document will not be available during external examinations.

\newpage
\section{System Fundamentals}

\subsection{Systems in organization}

\subsubsection{Planning and system installation}

Stages through which the development of a new system passes
through.

It might generally look like: \\
\textbf{Analysis} $\rightarrow$ \textbf{Design} $\rightarrow$
\textbf{Implementation} $\rightarrow$ \textbf{Operation} $\rightarrow$
\textbf{Maintenance}

\textbf{Context for a new system} \\
A new system can be created to replace one that is inefficient, redundant or
outdated. The first stage is planning, which will often require a feasibility
study for the new system:
\begin{itemize}
    \item Technical feasibility: Is the technology sufficient to implement
        the system?
    \item Economic feasibility: Is the system cost effective
    \item Legal feasibility: Does the system adhere to relevant laws and
        regulations
    \item Operational feasiblity: Are existing practices and procedures
        sufficient to support the system?
    \item Schedule feasibility: How long will the new system take?
\end{itemize}

\textbf{Change management} \\
Change management is the process of shifting individuals, teams,
departments, and organizations from the present to a desired state.

\textbf{Compatibility issues} \\
Legacy system: Old technology which is now out-dated, but possibly still in
use by a system due to various reasons such as being difficult to replace due
to complexity or using data which is incompatible with new data formats etc.

Business merger: the combining of two or more business entities. Has benefits such
as reduced costs but presents challenges such as ensuring both systems are
compatible.

\textbf{SaaS} \\
Software-as-a-Service (SaaS): A software delivery method where the software is
hosted remotely, and users pay to access it on a subscription basis.

\textbf{System installation processes} \\
Implementation methods for moving from old to new system:
\begin{itemize}
    \item Parallel: both systems work in parallel for a short time, until the
        transition to the new one is finished.
    \item Direct: The old one is halted, and the new one is immediately put
        into place.
    \item Pilot: the new one is implemented in a smaller part of the
        organization, and slowly extended over time.
    \item Phased: parts of the old system are replaced by parts of the new
        system one by one gradually.
\end{itemize}

\textbf{Data migration} \\
Data migration: transfer of data between formats, storage types or computer
systems.

Issues with data migration include compatibility issues and data loss.

\textbf{Testing} \\
Functional testing: the specific functions of a program are tested.

Data testing: normal, abnormal and extreme data is provided to the system.

Alpha testing: testing before the software is made public.

Beta testing: tested by initial users and feedback is obtained.

Dry-run testing: conducted using pen-and-paper by the programmers.

Integration testing: Entire system is tested

User acceptance testing: system is tested to see if it satisfied the customers'
needs.

Debugging: process of finding and correcting bugs in programs.

Validation: evaluating whether data input follows appropriate specifications.

Verification: ensuring data input is the same as the original source data.

\subsubsection{User focus}

\textbf{Documentation} \\
Internal documentation: documentations embedded within the source code itself
(comments, types, etc.)

External documentation: usually written as a separate document from the
program.

\textbf{Providing documentation}
\begin{itemize}
    \item Manuals
    \item Email support
    \item Embedded assistance or integrated user assistance
    \item Frequently Asked Questions
    \item Live chat sessions
    \item Online web portals
    \item Remote desktop connections
\end{itemize}

\textbf{User training}
\begin{itemize}
    \item Self-instruction
    \item Formal classes
    \item Online training
\end{itemize}

\subsubsection{System backup}

\textbf{Data loss} \\
Causes:
\begin{itemize}
    \item Accidental deletion
    \item Administrative errors
    \item Poor data storage system
    \item Building fires
    \item Natural disasters
    \item Viruses/malware
    \item Data corruption
    \item Firmware corruption
\end{itemize}

Prevention:
\begin{itemize}
    \item Regular backups
    \item Firewall/Anti-virus installation
    \item Offsite storage
    \item Removed hard copies
    \item Failover system: A system that, upon error or failure, is able to switch over
        to a spare/backup system and continue functioning.
\end{itemize}

\subsubsection{Software deployment}

\textbf{Software releases and updates}
\begin{itemize}
    \item Patches: updates to fix bugs and issues
    \item Upgrades: Novel functionalities or features as well as bug fixes.
    \item Updates: Minor upgrade
    \item Release: Finalised working version of software.
\end{itemize}

\subsection{System design basics}

\subsubsection{Components of a computer system}

\begin{itemize}
    \item Hardware: physical elements.
    \item Software: instructions that can be understood and executed by the
        CPU.
    \item Peripheral device: Auxiliary device that can communicate with a
        computer.
    \item Computer network: Interconnected computer systems.
    \item Human resources: people.
\end{itemize}

\textbf{Roles of a computer}
\begin{itemize}
    \item Client: recieves data from a server
    \item Server: sends data to a client
    \item Dumb terminal: consists of keyboard, monitor, network card.
    \item Thin client: low performance terminal.
    \item Email server: manages flow of email in and out of network.
    \item Router: accepts incoming data packets, reads their destination address, and
        distributes them across the network according to a routing policy.
    \item Domain Name System (DNS) server: attributes names to network addresses and
        resolves names by assigning them to the appropriate network entity.
    \item Firewall: hardware or software network infrastructure that controls and
        limits data flow access among entities. Used to offer protection.
    \item Client-server: architecture where client and server are connected.
\end{itemize}

\textbf{Social and ethical issues}
\begin{itemize}
    \item Reliability
    \item Integrity
    \item Inconsistency
    \item Security
    \item Privacy
    \item Anonymity
    \item Intellectual property
    \item Digital divide and equality of access
    \item Surveillance
    \item Global and cultural diversity
\end{itemize}

IT policies: enforceable procedures to ensure appropriate and secure use of a
computer system.

Standards and protocols: technical rules hardware and software should follow.

People and machines: Internet addiction is a social impact. Use of AI in
military or law-enforcement situations.

Digital citizenship: appropriate behavior in a digital world.

\subsubsection{System design and analysis}

Stakeholders: individuals, teams or organizations with an interest in the
project/system/etc.

\textbf{Obtaining requirements from stakeholders}
\begin{itemize}
    \item Structured/Unstructured interviews
    \item Closed/Open questionnaires
    \item Direct observation.
\end{itemize}

Gathering information to arrive at a workable solution: examining current
systems, examining competing products.

Requirements Specification Document: specifies requirements.

\textbf{Illustrating system requirements}
\begin{itemize}
    \item Flowcharts
    \item Data-flow diagrams
    \item Structure charts
\end{itemize}

Prototyping: working or non-working preliminary version of product.

Iteration: repeating the process multiple times until arriving at the desired
point.

\subsubsection{Human interaction with the system}

\begin{itemize}
    \item Accessiblity: potential of a product to meet the needs of as many
        individuals as possible.
    \item Usability: potential of a product to accomplish user goals.
    \item Ergonomics: design of safe and comfortable products.
\end{itemize}

\textbf{Components of usability}
\begin{itemize}
    \item Simplicity
    \item Effectiveness
    \item Efficiency
    \item Error
    \item Learnability
    \item Memorability
    \item Readability
    \item Satisfaction
\end{itemize}

% =============================================================================
% ====== Computer Organization ================================================
% =============================================================================
\section{Computer Organization}

\subsection{Computer organization}

\subsubsection{Computer architecture}

\begin{itemize}
    \item Control Unit (CU): retrieves instructions from primary memory and
        handles the sequence of their execution.
    \item Arithmetic Logic Unit (ALU): performs all basic arithmetic, logic,
        input/output operations.
    \item Memory Address Register (MAR): holds memory address of data to be
        used by the ALU
    \item Memory Data Register (MDR): holds data to be used by the ALU.
\end{itemize}

\textbf{Primary memory}
\begin{itemize}
    \item Cache: a small amount of fast temporary memory placed between CPU
        and main RAM, used as cache. Separated into L1 and L2 caches.
    \item Random Access Memory (RAM): temporary data 
    \item Read-only memory (ROM): permanent data
\end{itemize}

\textbf{Instruction cycle} \\
Known as the fetch-decode-execute cycle:
\begin{enumerate}
    \item Fetch instructions from primary memory to CU
    \item Decode instruction in CU
    \item Execute instruction
    \item Store result of execution and check for next instruction
\end{enumerate}

\subsubsection{Secondary memory}

Secondary memory is a slower type of memory for data storage that can be both
read to and written from. It is non-volatile. Examples are USBs, hard drives,
etc.

\subsubsection{Operating and application systems}

\textbf{Functions of an OS}
\begin{itemize}
    \item Peripheral communication
    \item Memory management
    \item Resource monitoring and multi-tasking
    \item Networking
    \item Disk access and data management
    \item Security
\end{itemize}

\textbf{Software applications} \\
Software applications: word processors, spreadsheets, database systems, etc.

GUI: graphical user interface

\subsubsection{Binary representation}

bit: basic unit of information, value of either 0 or 1.
\begin{align*}
    1 \text{ byte} = 8 \text{ bits}
\end{align*}

Number systems: decimal - base 10, binary - base 2, hexadecimal - base 16.

Convert from decimal to binary: repeatedly divide by 2 and keep track of the
remainders (which will be 0 or 1). The reverse order of the remainders obtained
is the number in binary.

Covnert from binary to decimal: multiply each digit with its respective
multiplier and sum the results.

\subsection{Negatives in binary}
Most significant bit (MSB): the bit having the largest value in a binary
number.

Least significant bit (LSB): the bit having the smallest value in a binary
number.

Negative numbers (Two's complement): used to represent signed binary numbers.
To find a number's two's complement representation, first write the absolute
value in binary, then invert the bits, then add 1. The largest number possible
in this system is 127, while the smallest is -128.

Negative numbers (Sign and magnitude representaiton): the left-most bit
represents the sign, while the rest carries the magnitude.

Binary fractions (fixed-point representation): the number is divided at a
chosen fixed point. the part to the left of it is the integer part while the
right is the fractional part. To the left of the fixed point are increasing
powers of two as per usual, and to the right are decreasing powers, just like
in the normal decimal system.

Hexadecimal: base-16 system.

ASCII: character-encoding scheme for English alphabet.

Strings: sequence of characters.

Colours: can be represented using hexadecimal RGB colour values - a six-digit
hexadecimal number where the first two digits represent red, next two green,
last two blue.

% TODO
\subsubsection{Simple logic gates}
[INCOMPLETE]

\section{Networks}
Definitions:
\begin{itemize}
    \item bus topology: computer network in which a `bus' connects all devices
        together through a common cable.
    \item cable: wire or fibre used to connect computers in a network.
    \item check digit: extra digit added to numerical data that is used to
        check for data integrity after data has been transmitted, stored, input
        or processed.
    \item data integrity: accuracy of data
    \item check sum: error-detecting procedure that generates a sum from the
        digits of a number.
    \item data packet: portion of a message that is transmitted through a
        network. contains data such as check digits and destination address.
    \item gateway: link residing between computer networks and is responsible
        for converting data passing through into the appropriate format.
    \item handshaking: exchange of predetermined signals to signify that a
        connection has been established.
    \item hub: network connection point for devices. data arriving at a hub is
        copied and sent to all connected devices.
    \item switch: like a hub, but can identify which device is connected to
        which port, and then transmit data to specific port/device as needed.
    \item router: connects multiple networks together to form larger networks
    \item ISDN: integrated services digital network - international
        communication standard allowing for transmission of audio/video/etc.
        over digital telephone lines.
    \item local area network (LAN): network where connected computers are
        within a limited geographical area.
    \item microwave transmission: electronic communication without cables.
    \item modem (modulator/demodulator): converts digital signals into audio
        signals and back. allows distant communication.
    \item network: interconnected computer systems, which can share data.
    \item packet: group of bits.
    \item packet switching: network communication method that create and
        transmits small units of data through a network, independetly of the
        overall message.
    \item parity bit: error-detecting procedure that appends binary digits to
        a group of binary digits. the sum of the digits, including the appended
        digit, establishes the accuracy of the data.
    \item protocol: international rules that ensure the transfer of data
        between systems.
    \item TCP/IP: protocol used to connect hosts to the internet.
    \item wide area network (WAN): larger geographical area than LAN.
\end{itemize}

% =============================================================================
% ========= Networks ==========================================================
% =============================================================================
\subsection{Networks}

\subsubsection{Network fundamentals}

\textbf{Types of networks}
\begin{itemize}
    \item local area network: connects devices within a limited geographical
        area, such as an office building or school. one advantage is it allows
        sharing of peripheral devices (like printers) between devices.
    \item wireless local area network: like LANs, but wireless. Being wireless
        is convenient, but one disadvantage is that it is also much less
        secure compared to LANs.
    \item virtual local area network: this is used to partition LANs into
        \textit{logical separate networks}. For example, different departments
        within the same office might have their own VLAN, allowing the network
        to be more organized without much costs.
    \item wide area network: connects devices across a large geographical area.
        The internet is a WAN.
    \item storage area network: a network created for accessing large storage
        devices from servers more conveniently.
    \item intranet: broad term for a collection of private networks utilizing
        protocols like TCP/IP. they can be thought of as a more private
        internet.
    \item internet: connects a large number of networks together. largest WAN
        in the world. it is also a decentralized network. no one entity
        controls the internet.
    \item extranet: utilizes the internet to allow \textbf{controlled access}
        by specific users to a specific LAN or WAN.
    \item internet of things: network of `things' that are able to connect to
        the internet, e.g. refrigerators, cars, etc.
    \item virtual private network: allows clients from remote locations to
        connect to the network and appear to be inside the LAN/WAN as if they
        were physically present. it is much more secure and can bypass
        geographical restrictions.
    \item personal area network: much shorter range; connects devices centered
        around one individual, e.g. your mobile and drawing tablet and printer
        and camera and so on.
    \item peer-to-peer: a network which does not utilize the client-server
        model. Instead, it is a distributed network where all systems, called
        nodes or peers, act as both the client and the server at the same time,
        consuming and supplying resources from and to other devices at the
        same time.
\end{itemize}

\textbf{Importance of standards} \\
Standards enable compatibility through a common `language' and predefined rules
that all parties agree upon.

\textbf{Communication over networks} \\
Open Systems Interconnection Model, established by the International Standards
Organization (ISO). Contains seven layers aimed at facilitating communication
across systems:
\begin{itemize}
    \item Physical: transmits binary data over media between devices. e.g
        RS232-C.
    \item Data link: error handling of physical transmission, flow control and
        handling transmission rates. e.g Ethernet
    \item Network: handles routing of packets across a network. e.g IP
    \item -- (Above: Physical communication) -- (Below: Virtual communication)
    \item Transport: end to end connection (between hosts). this is where data
        segments are defined, data transfer occurs, and then the data is
        reassembled at its destination. e.g TCP
    \item Session: managing sessions between two users. e.g windows system core
        protocol.
    \item Presentation: data format information, compression information,
        encryption information, etc. e.g Portable Network Graphics (PNG)
    \item Application: performs services consumed by end users. e.g HTTP
\end{itemize}

TCP/IP Model: Similar to the OSI model, describes functions taking
place at each layer of protocols within the TCP/IP suite:
\begin{itemize}
    \item Network Access: layer 1 and 2 of OSI
    \item Internet: layer 3 of OSI
    \item Transport: layer 4
    \item Application: layers 5, 6 and 7
\end{itemize}

\textbf{Virtual Private Networks (VPNs)} \\
All traffic on VPNs is encrypted, authenticated and then sent along virtual
tunnels. Secure VPN technologies include the internet protocol security
protocol (IPSec protocol), the Secure Sockets Layer, or Transport Layer
Security with encryption.

Common VPN types:
\begin{itemize}
    \item Site-to-site VPN: Connects entire networks together through gateways,
        for secure data interchange.
    \item Remote-access VPN: Connects individual hosts to private networks.
        Every host has a VPN client software installed.
\end{itemize}

\textbf{Use of a VPN} \\
Benefits:
\begin{itemize}
    \item Easier communication
    \item Secure connections
    \item Remote access
\end{itemize}

\subsubsection{Data transmission}

\textbf{Necessity of Protocols} \\
Protocols are important because they define rules about message/data formats,
error checking methods, data compression and ecnryption/decryption and so on.

Protocols also guarantee:
\begin{itemize}
    \item Source integrity: identity of the sender has been validated
    \item Flow control: Controlling the rate of data transmission to prevent
        overload of resources while also maintaining efficiency
    \item Congestion management: Handling congestions which occur when requests
        to the network exceed the offered capacity
    \item Deadlock prevention: preventing situations where two or more networks
        are waiting for the other to finish the process, thus neither is able
        to finish.
    \item Error checking and correction: determining if an error has occurred,
        and recovering data lost is possible.
\end{itemize}

\textbf{Factors that affect speed of data transmission}
\begin{itemize}
    \item Bandwidth of network
    \item Data transfer rate of storage devices
    \item Interferences
    \item Malicious software
    \item Packet loss and retransmission
\end{itemize}

\textbf{Data compression}
\begin{itemize}
    \item Lossy: some loss of information is acceptable, and there is no way to
        retrive the original data.
    \item Lossless: reduces the number of bits by identifying redundant data
        and encoding it in a more compact format to save space without losing
        the original data.
\end{itemize}

\textbf{Packet switching} \\
A data packet is a unit of information in a form suitable for transferring
between computers.

Packet switching is a method where data is grouped into packets. The original
file is divided into packets, each of which may follow a different path to the
destination.

\subsubsection{Wireless Networking}

Advantages:
\begin{itemize}
    \item Less hassle and possibly costs
    \item Easy to access
\end{itemize}
Disadvantages:
\begin{itemize}
    \item Relatively low speed
    \item Less secure and higher error rates
\end{itemize}

% TODO
\textbf{Hardware components of a wireless network}
\begin{itemize}
    \item [INCOMPLETE]
\end{itemize}

% TODO
\textbf{Software components of a wireless network}
\begin{itemize}
    \item [INCOMPLETE]
\end{itemize}

\textbf{Methods of network security}
\begin{itemize}
    \item Passwords
    \item Antivirus program
    \item Firewall (software/hardware)
\end{itemize}

\textbf{Encryption} \\
Encryption: Uses mathematical algorithms and encryption keys to protect data.
In symmetric-key encryption, the same key is used for encryption and
decryption. Public-key encryption uses a public key which is available to
everyone for encryption, and a private key for decryption. Both are
mathematically linked. This is used on the internet, such as TLS and SSL
protocols.

% =============================================================================
% ====== Coputational thinking ================================================
% =============================================================================
\section{Computational Thinking}

\subsection{General principles}

\subsubsection{Algorithms}
Algorithms are a series of instructions designed to solve a problem.
Algorithms can be expressed in different ways, for example with natural
language, flow charts or diagrams, pseudocode or programming languages.

Algorithms are (often) procedural, that is, they follow a set of given
instructions one by one. They may also include sub-procedures, which break the
problem into smaller sub-problems and solve them one by one, in order to more
easily solve the original problem at hand.

Algorithms can consist of conditionals for decision-making, iteration
constructs for repeating tasks, inputs and outputs for manipulating or
consuming data, pre-conditions (data and information needed before a procedure
is called) and post-conditions (after a procedure finishes), exceptions (events
that might disrupt the flow of the algorithm's execution) and so on.

\textit{Concurrency} means something happening at the same time as something
else. In terms of computation, it refers to instructions being processed at
the same time. Here, procedures are broken into sub-procedures that can be
handled concurrently by processors. This helps efficiency but requires more
planning and coordination.

\textit{Abstraction} refers to when unnecessary low-level implementation
details of something are hidden, and only the relevant and fundamental parts
of it are considered.

\textit{Object-oriented programming} uses abstraction by representing entities
as programming objects. These objects describe the data/properties and
behavior/methods of the entity it represents.

\textit{Variables} are identifiers that can be used to reference values stored
in the computer, during program execution.

\subsection{Connecting computational thinking and program design}

\subsubsection{Linear arrays}
Linear arrays are one-dimensional arrays that can store multiple
values, unlike variables.

\textit{Sequential search:} goes through each item one by one until the search
is successful or there are no more items.

\textit{Binary search:} works on sorted lists by dividing them into half at
each iteration and recursively searching through the sublist containing larger
or smaller elements depending on the search value

\textit{Bubble sort:} sorting algorithm; repeatedly steps through the array,
compares adjacent elements and puts them in order. makes multiple passes until
no swaps are necessary and the array is sorted. slow and impractical algorithm.

\textit{Selection sort:} sorting algorithm; divides the input array into two
sub-arrays, one containing the sorted elements, the other unsorted elements. At
first, the size of the first sub-array is 0. The algorithm goes through the
unsorted array looking for the max/min element and places it at the leftmost
position, next to the sorted array, and then extends the borders of the sorted
array. it then repeats until the size of the unsorted array is 0.

\subsubsection{Collection operations}
\begin{itemize}
    \item addItem(): add an item to the collection
    \item getNext(): returns items from the collection, starting from the first
        and then the next items upon repeated calls of this method.
    \item resetNext(): resets the getNext() function to go back to the
        beginning.
    \item hasNext(): returns true if the collection has items left that have
        not yet been returned by getNext(), otherwise false.
    \item isEmpty(): whether the collection has items or not
\end{itemize}

\subsubsection{Fundamental operations of a computer}
\begin{itemize}
    \item LOAD: load a value from RAM into the CPU's control unit
    \item STORE: store a value from the control unit into the RAM
    \item ADD: adds two numbers together
    \item COMPARE: compare two numbers together
\end{itemize}

Compound operations are made up of many fundamental operations, to do more
complex tasks.

\subsection{Introduction to programming}

\subsubsection{Features of a programming language}
A programming language is a combination of its semantics (the meaning of every
construction possible in the language) and syntax (the structure of the
language). A grammar is a meta language used to define the syntax of the
language. Languages also have a specific vocabulary, i.e keywords that have
special meanings. Commands in computer languages are always unambiguous.

\subsubsection{Higher-level languages}
Higher-level languages are necessary because they make program development
simpler, faster and more understandable. However, these languages cannot be
executed by a CPU which only understands simple fundamental instructions.
Thus we use compilers and interpreters:
\begin{itemize}
    \item \textit{Compiler:} translates from high to low level only once, and
        produces an output file that can be directly executed by the CPU.
    \item \textit{Interpreter:} reads the high-level instructions, analyzes,
        transaltes and executes them on the fly. This happens every time the
        program needs to be run.
\end{itemize}
Some languages are compiled to bytecode, which is a more simpler and compact
set of instructions compared to high-level languages but still not as low-level
as direct CPU instructions. This bytecode is executed by a virtual machine
program designed for that bytecode.

\subsubsection{Common features of a language}
\begin{itemize}
    \item Variable: stores a data element
    \item Constant: stores data that does not change
    \item Operator: performs operations on data types
    \item Object: made up of data and methods that describe a certain entity.
\end{itemize}

\subsubsection{Advantages of using sub-programs}
\begin{itemize}
    \item Code reuse
    \item Breaking down problem into easier sub-problems
    \item Maintainability and traceability
    \item Abstraction
\end{itemize}

% =============================================================================
% ====== Abstract Data Structures =============================================
% =============================================================================
\section{Abstract Data Structures}

\subsection{Abstract Data Structures}

\subsubsection{Thinking recursively}

\textbf{Recursion} is when a method calls itself, until some terminating
condition is met. It does not use looping constructs (like for or while loops).

It follows the basic problem solving principle of breaking a problem down into
smaller sub-problems and solving them.

A recursive algorithm can be written using looping constructs, and vice versa.

Examples of recursive algorithms: Tower of Hanoi problem, Koch snowflakes.

\subsubsection{Two dimensional arrays}

A one-dimensional array can be considered a single line of elements. A two-
dimensional array would then be a \textit{table} of elements.

2d-arrays are indexed by two subscripts: array[y][x]. If we think of the
2d-array as a plane, then the y subscript indicates the position of the element
along the vertical axis (starting from 0 at the top), while the x subscript
indicates the position along the horizontal axis (starting from 0 at the left).
The top-left element would thus be array[0][0], while the bottom right would be
array[n-1][n-1] fora 2d array of size n by n.

\subsubsection{Stacks}

Stacks are a last-in, first-out (LIFO) data structure, meaning that the last
element inserted into the stack is the first element retrieved from it.

Stacks have three important methods:
\begin{itemize}
    \item push(): adds an item onto the stack
    \item pop(): removes the last item added to the stack
    \item isEmpty(): checks if the stack has any elements or not
\end{itemize}

Applications:
\begin{itemize}
    \item Back button of a web browser: the last page visited is the first one
        the back button points to. as we keep pressing it, the browser keeps
        going back to the previous pages opened. thus it follows a LIFO
        structure
\end{itemize}

\subsubsection{Queues}

Queues are a first-in, first-out (FIFO) structure.

The three queue methods are:
\begin{itemize}
    \item enqueue(): adds an item to the end of the queue
    \item dequeue(): removes an item from the start of the queue
    \item isEmpty(): checks if queue is empty or not
\end{itemize}

Applications:
\begin{itemize}
    \item Used to model physical queues
\end{itemize}

\subsubsection{Linked lists}

Each node in linked lists contains two items: one is the data and the other is
a pointer to the next element. A header node is used at the beginning which
only contains one item, the pointer to the next node. The last node's pointer
points to a null value to indicate the end of the list.

To remove a node, simply change the pointer of the previous node to the next
node instead.

To insert a node in some position, change the pointer of the previous node to
the new node, and set the pointer of the new node to the next node.

\subsubsection{Double linked lists}

In double linked lists, each node has two pointers - one points to the previous
element, and one points to the next element.

\subsubsection{Circular linked lists}

In circular linked lists, the last node does not point to a null value, but
rather back to the header node (or the first node if there is no header node).

\subsubsection{Trees}

Definitions:
\begin{itemize}
    \item Children: the nodes below a given node
    \item Key: the data field of a node
    \item Leaf: a node with no children
    \item Level: distance of a particular node from the root node
    \item Height: number of edges from the top node to the deepest leaf
    \item Parent: the node above a given node
    \item Path: sequence of nodes travelled to get from one to the other
    \item Root: the node at the top of the tree, with no parents
    \item Subtree:
    \item Traversing: to visit all the nodes of the tree in some order
    \item Visiting: arrive at a node and perform some operation on it. if the
        algorithm just passes over the node then it is not considered a visit
\end{itemize}

\textbf{Inorder traversal:} The algorithm calls itself on the left subtree,
then visits the current node, then calls itself on the right subtree. (LCR)

\textbf{Preorder traversal:} The algorithm visits the current node, then calls
itself on the left subtree, then calls itself on the right subtree. (CLR)

\textbf{Postorder traversal:} The algorithm calls itself on the left subtree,
then calls itself on the right subtree, then visits the current node.

An easy way to remember is that the namer of the traversal tells you when the
current node is visited, otherwise the algorithm first calls itself on the left
subtree, and then the right one. For inorder, the current is \textit{in}
between the other positions, L-C-R. For pre, it is \textit{before} the others,
so C-L-R. For post, it is \textit{after} the others, so L-R-C.

\textbf{Operations on binary trees}

Binary trees are sorted, meaning that values to the left are smaller and values
to the right are larger. This is useful for determining the result of adding
or removing nodes from a tree: add/remove the node in its correct position.

\subsubsection{Static and dynamic data structures}

In static structures, the size of the structure is constant and pre-determined.
Memory allocation is fixed and the space reserved will always be available.
Since the size is predetermined, there is also less overhead and hence faster
speed.

In dynamic structures, the size changes depending on how many items are being
stored. They are generally slower due to overhead of constantly checking
memory allocation and size to make sure there is enough space. However, they
are more space efficient since only the required amount of memory is allcoated.

% =============================================================================
% ====== Resource Management ==================================================
% =============================================================================
\section{Resource Management}

\subsection{Resource Management}

\subsubsection{Critical resources}

\textbf{Primary memory:} This is connected directly to the CPU which uses it to
store instructions. It includes RAM of which there are two types: static and
dynamic. SRAM is more expensive, but retains data as long as there is power.
Data in DRAM gradually leaks and needs to be refreshed preiodically. Read-only
memory (ROM) is also primary memory which is slower than RAM. It holds critical
information needed to boot the PC. It usually holds the Basic IO System (BIOS)
and sometimes the entire OS as well.

\textbf{Secondary storage:} these are low cost, have larger capacity and
permenent storage. There are two broad types: direct access (USBs, DVDs), where
data can be accessed, retrieved, and stored \textit{directly} without having
to read through all the previous data. The other type is sequential acccess
(magnetic tapes), where previous data has to be read sequentially until the
requested data can be located.

\textbf{Processor speed:} MIPS (million instructions per second) is used to
measure processor speed. Clock rate refers to the frequency at which the
processor runs, and can be a rough indicator of processor performance. It is
usually measured in gigahertz.

\textbf{Bandwidth:} Memory bandwidth is the rate at which data can travel from
SRAM to DRAM to the processor and vice versa. It is expressed in MB/s. Peak
theoretical bandwidth is the theoretical limit, while sustained memory
bandwidth is usually less and affected by various factors.

\textbf{Screen resolution:} number of pixels, height times width. The more
pixels, the better the resolution.

\textbf{Disk storage:} e.g. hard disk drive, solid state drives, etc.

\textbf{Sound processor:} Sound cards are used to process audio.

\textbf{Graphics processor:} these are massively parallel processors which are
very efficient at manipulating and processing graphics and images.

\textbf{Network connectivity:} A Network Interface Card is used by devices for
network connections.

\subsubsection{Availability of resources}

\textbf{Mainframes:} used by large organizations to handle large-bandwidth
communication, bulk data processing, etc.

\textbf{Supercomputers:} very fast and expensive, focused on mathematical
calculations. Performance measured in floating-point operations per second
(FLOPS).

\textbf{Other devices:} servers, PCs, tablets, digital cameras, etc.

% TODO
\subsubsection{Limitation of resources}
[INCOMPLETE]

% TODO
\subsubsection{Problems due to insufficient resources}
[INCOMPLETE]

% TODO
\subsubsection{Role of the Operating System}

\textbf{Managing memoery and processes:} [INCOMPLETE]

\textbf{Managing peripherals:} [INCOMPLETE]

\textbf{Scheduling:} [INCOMPLETE]

\textbf{Policies and account management:} [INCOMPLETE]

\textbf{Interrupts:} [INCOMPLETE]

\textbf{Polling:} [INCOMPLETE]

\textbf{Dedicated OSs:} [INCOMPLETE]

\textbf{OS and complexity hiding:} [INCOMPLETE]

% =============================================================================
% ====== Control ==============================================================
% =============================================================================
\section{Control}

\subsection{Control}

\subsubsection{Centralized control systems}

\textbf{Control system:} one or more devices that guide other devices or
systems, allowing for complete automation of tasks. They may contain sensors
to gain feedback from the environment and motors to control actuators in an
appropriate manner. An example is automatic doors in supermarkets, which have
sensors to detect when someone is entering or leaving and send signals to a
microprocessor. This allows the operating system to operate the actuators and
open/close doors.

Control systems are mostly input, process and output systems. An input is
provided to some algorithm, which performs some action and provides an output.
This output can then be measured again by the control system and provided back
as input.

Examples:
\begin{itemize}
    \item Heating systems: an initial temperature is taken as input, which 
        indicates the ideal value of the output and the `goal' of the system.
        Feedback is used heavily. Sensors measure the temperature, the
        processor decides what action to perform to reach the goal, the output
        transducers change the temperature accordingly. The sensor again takes
        the new temperature as input, and the loop continues.
    \item Taxi meters:
    \item Elevators: takes the desired destination as input and processor
        decides how to operate machinery to reach the destination. other
        inputs may be taken such as current location, location of other
        elevators, load on the elevator etc.
    \item Washing machines: sensors determine and control the load size in the
        washing machine, water level, temperature, and the user interface
        (buttons or a touch screen interface). The processor takes this
        information and decides how to operate the machinery. actuators are
        used to turn motors and operate the washing machine.
    \item Process control: used to maintain the output of a specific process
        within a desired range. open and closed loop controllers are used. For
        example, a tank with liquid in it and a coil for heating the liquid.
        An open feedback loop would need to know the relationship between the
        heat dissipated and the temperature increase of the liquid to determine
        the liquid's temperature. In closed loop, the temperature of the liquid
        would be detected using temperature sensors, and the coil temperature
        adjusted accordingly to achieve the desired liquid temperature.
    \item Device drivers: computer programs for controlling computer devices,
        such as keyobards and printers and graphics cards. these let the OS
        access the device through a layer of abstraction.
    \item Domestic robots: e.g. automatic vacuum cleaners. a number of sensors
        are used within these to gather information about their surroundings
        and perform their task.
    \item GPS systems: a large complex computer controlled system. a GPS
        receiver includes a sensor that locates multiple GPS satellites in
        space, figures out the distance from each, calculates the time
        difference and uses a process called `trilateration' to deduce its
        location. also includes inputs from users to pinpoint desired
        locations.
    \item Traffic lights: as discussed before, traffic lights can be
        fixed time, which are open loop control systems in that they dont
        receive feedback from the environment. the are pre-configured to change
        at intervals. dynamic control traffic lights use a close loop control
        system, that is they gather feedback from the environment, to
        dynamically control traffic, based on some appropriate algorithm. they
        may use mounted sensors, located on the traffic lights, or embedded
        sensors in the surface of the road.
\end{itemize}

\textbf{Use of microprocessors and sensor inputs} \\
\begin{itemize}
    \item General purpose: capable of running a wide range of programs, usually
        integrated into a larger system, e.g. CPUs in desktop computers
    \item Embedded (micro)controller: standalong chips designed to perform some
        specific task, and do not need an entire computer system. found in many
        control systems described above. generally need less power and smaller
        in size as well.
    \item GPU: special kind of microprocessor for graphics and images.
\end{itemize}

The main advantage of microprocessors in control systems is that process data
much faster than humans could and can react to changes quickly. They are also
automated and hence less error-prone, and can operate with little off-time.

The main disadvantage is that they are pre-programmed and cannot handle
unexpected situations, which may be common in some systems. They also cannot
function if there is a power shortage.

\textbf{Input sensors} convert physical quantities such as speed, temperature
to discrete digital quantities using an analog-to-digital converter. These can
then be read by microprocessors. Sensors share the following properties that
determine their quality:
\begin{itemize}
    \item Accuracy of measurement for the physical quantity
    \item Range of measurement for the physical quantity
    \item Resolution (the smallest increment detectable) for the physical
        quantity.
\end{itemize}

\textbf{Different input devices for data collection} \\
Sensors are used as input devices. There are many types of sensors:
\begin{itemize}
    \item sound
    \item motion
    \item vibration
    \item image/optical: active pixel sensors (for cameras), infrared (alarm
        systems)
    \item pressure
    \item temperature
    \item proximity
\end{itemize}

\textbf{Transducers} are devices that convert one form of energy into another.
In computer systems, they are responsible for converting physical quantities
into electrical signals and vice versa. This process is called transduction.
Sensors can also be categotized as transducers since they detect physical
quantities and convert them into digital quantities.

\textbf{Actuators} are another type of transducers, used when returning the
output of the processor. They are responsible for moving some kind of
mechanism. They take some form of energy, usually electric current, and convert
it into motion.

Thus a control system may look like: \\
Input $\rightarrow$ Transducer (sensor) $\rightarrow$ Processore $\rightarrow$
Transducer (actuator) $\rightarrow$ Output

\textbf{Role of feedback} \\
Feedback is the process wher information from the result of an output is used
as new input to the control system, creating a sort of loop.

It is important in control systems that react to their environment and its
changes. It helps keep the system stable.

For example, a dynamic traffic light which uses feedback: first, the motion
sensor detects that there are no cars (input), the processor decides to turn
the light red and allow pedestrians to pass (output). After a while, a line of
cars forms since the light is red. The system again detects (2nd input) this
and now turns the light green to allow cars to pass (2nd output). This is an
example of how feedback from one output is used to determine the next input
and action, allowing the system to remain in a stable state.

% TODO
\textbf{Social impacts and ethical considerations}
\begin{itemize}
    \item Electronic tagging: [INCOMPLETE]
    \item Surveillance: [INCOMPLETE]
    \item Safety systems: [INCOMPLETE]
\end{itemize}

\subsubsection{Distributed systems}

\textbf{Centralized system:} all resources reside in a single system. clients
need to be able to connect to this system. Main benefit is lower operational
costs, greater security, less administrative overhead and backup complexity.

\textbf{Distributed system:} resources are distributed across various systems
and each system is self-sustaining. a central system may coordinate distributed
ones, but each can still function on its own, and failure of one distributed
system will not be fatal to the overall system. these systems are more scalable
and fault-tolerant.

\textbf{Agents:} these can by anything that perceives its environment, through
sensors and acts upon it through effectors.

\textbf{Autonomous agents:} these are entities acting on behalf of an owner,
with a degree of autonomy and little to no interference from the owner. Their
environment often plays a significant role in their function:
\begin{itemize}
    \item Accessible/Inaccessible: the agent may or may not be able to obtain
        compelte and accurate information about the environment.
    \item Deterministic/Non-deterministic: whether the agent's actions will
        have a pre-defined, guaranteed effect. The next state of the
        environment depends entirely on its initial state.
    \item Episodic/Non-episodic: whether an agent's actions are divided into
        `episodes' or not. Episodes are independent, i.e. actions in previous
        episodes do not affect actions in subsequent episodes.
    \item static/dynamic: the environment may change while the agent is
        deliberating an action
    \item discrete/continuous: the environment may have discrete, clearly
        defined states or continuous states.
\end{itemize}

\end{document}

