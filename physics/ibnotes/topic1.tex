% ================================
\subsection{Measurements in Physics}

\subsubsection{Fundamental and derived SI Units}

SI Units: The international system of units.

Fundamental Units: These are the 7 basic units all other SI units are made up
of. They are:
\begin{center}
    \begin{tabular}{ |c|c|c| }
        \hline
        \textbf{Name} & \textbf{Symbol} & \textbf{Quantity} \\
        \hline
        Meter         & m               & distance \\
        Kilogram      & kg              & mass \\
        Second        & s               & time \\
        Ampere        & A               & electric current \\
        Kelvin        & K               & temperature \\
        Mole          & mol             & substance \\
        Candela       & cd              & luminous intensity \\
        \hline
    \end{tabular}
\end{center}

Derived units: These are all the units derived from the fundamental ones. For
example, ms$^{-1}$ for speed.

\subsubsection{Scientific Notation and metric multipliers}

Numbers in the form $a \times 10^b$ are said to be in scientific notation form.
Here, $b$ must be a positive or negative integer, and $1 \leq a \leq 10$.

Metric multipliers are used to express quantities using powers of 10. These
can be added as prefixes to basic units. Common ones are:
\begin{center}
    \begin{tabular}{ |c|c|c| }
        \hline
        \textbf{Power} & \textbf{Prefix} & \textbf{Symbol} \\
        \hline
        10$^{-9}$      & nano-           & n \\
        10$^{-6}$      & micro-          & $\mu$ \\
        10$^{-3}$      & milli-          & m \\
        10$^{-2}$      & centi-          & c \\
        10$^{-1}$      & deci-           & d \\
        10$^3$         & kilo-           & k \\
        10$^6$         & mega-           & M \\
        10$^9$         & giga-           & G \\
        \hline
    \end{tabular}
\end{center}
More can be found in the data booklet.

Significant figues: The number of digits used to express a quantity/number. It
tells us how precisely we know that number.

Rules for significant figures:
\begin{itemize}
    \item For integers, all digits count \textbf{except} leading and trailing
        zeros.
    \item For decimals, all digits count \textbf{except} leading zeros.
    \item For scientific notation, the number of significant figures in $a$
        are used.
    \item For multiplication or division, the least number of significant
        figures must be kept in the result.
    \item For addition or subtraction, the least number of decimal places must
        be kept in the result.
\end{itemize}
{\small (leading - zeros in the front, trailing - zeros at the end)}

\subsubsection{Orders of Magnitude}
Expressing a quantity as a power of 10 is called the order of magnitude for
that quantity. For example, the mass of the Milky Way galaxy has an order of
magnitude of $10^{41}$.

\subsubsection{Estimation}
Estimation.

% ================================
\subsection{Uncertainties and Errors}

\subsubsection{Random and Systematic Errors}

Random error: Unbiased uncertainty due to random fluctuations in measurements.
Also includes reading uncertainties (uncertainty due to precision of
instruments).

These cannot be eliminated but can be minimized by, for example, doing
multiple trials of an experiment and taking the average of recorded values.

Systematic error: Biased uncertainty due to error in methodology or broken
instrument.

\subsubsection{Absolute, fractional and percentage uncertainties}
\begin{itemize}
    \item Absolute: actual uncertainty in a measurement, expressed using the
        relevant units. For digital instruments, it is the least count. For
        analog, it is half of the least count.
    \item Fractional: ratio of absolute uncertainty to the measurement value.
    \item Percentage: fractional uncertainty as a percentage.
\end{itemize}
{\small (Least count: smallest value that can be measured with that instrument)}

Some rules for uncertainties in calculation:
\begin{itemize}
    \item For addition or subtraction, the absolute uncertainty of the result
        is the sum of the absolute uncertainties in all quantities involved.
    \item For multiplication or division, the fractional uncertainty of the
        result is the sum of the fractional uncertainties in all quantities
        involved.
    \item For powers and roots, the fractional uncertainty of the result is
        the fractional uncertainty of the quantity multiplied by the absolute
        value of the power.
\end{itemize}

\subsubsection{Error bars}
Those extra lines on a point to indicate uncertainty range. These are not shown
if the error is negligible.

\subsubsection{Uncertainty of gradient and intercepts}
To get uncertainty of gradient and intercepts, draw a line through the error
bars with the maximum gradient, and a line with the minimum gradient.
Uncertainty in the gradient is then $(max - min) / 2$ and similarly for the
intercepts.

% ================================
\subsection{Vectors and Scalars}

\subsubsection{Vector and scalar quantities}
\begin{itemize}
    \item \textbf{Vector:} quantities with magnitude and direction, e.g.
        velocity and acceleration.
    \item \textbf{Scalar:} quantities with just magnitude, e.g. time, mass.
\end{itemize}

\subsubsection{Combination and Resolution of Vectors}
Dealing with vectors is much easier after splitting them into their x and y
components, and vectors can also be reconstructed using their components:
\begin{itemize}
    \item x-component = $A\cos\theta$
    \item y-component = $A\sin\theta$
    \item $F = \sqrt{(F_x)^2 + (F_y)^2}$
    \item $\theta = \arctan\frac{F_y}{F_x}$
\end{itemize}
Be careful about the angles though, some situations can get confusing.

