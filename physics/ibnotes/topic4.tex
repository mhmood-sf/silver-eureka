% ================================
\subsection{Oscillations}

\subsubsection{Simple harmonic oscillations}
A restoring force, which tries to bring the system to its equilibrium position
is necessary for oscillations.

Simple harmonic motion is defined by the property: $a \propto -x$, i.e
magnitude of the acceleration is proportional to displacement and the direction
of the acceleration is towards the equilibrium position.

During these oscillations, $E = E_k + E_p = E_{kmax} = E_{pmax}$.

\subsubsection{Time period, frequency, amplitude, displacement, and phase
difference}
\begin{itemize}
    \item Time period: time taken to complete one full oscillation.
    \item Frequency: number of oscillations per second, $f = \frac{1}{T}$.
    \item Amplitude: maximum displacement from the equilibrium position.
    \item Displacement: distance from equilibrium position.
    \item Phase difference: amount by which a curve is shifted forwards
        relative to another. $\phi = \frac{shift}{T} \times 360^{\circ}$.
\end{itemize}

\subsubsection{Conditions for simple harmonic motion}
\begin{itemize}
    \item $a \propto -x$
    \item period and amplitude are constant
    \item period is independent of amplitude
    \item displacement, velocity, and acceleration are sine or cosine functions
        of time
\end{itemize}

% ================================
\subsection{Travelling waves}

\subsubsection{Travelling waves}
Wave: disturbance that travels in a medium (or a vacuum for electromagnetic
waves) transferring energy and momentum from one place to another. The
direction of propagation of the wave is the direction of energy transfer. There
is no large-scale motion of the medium itself as the wave passes through.

\subsubsection{Wavelength, frequency, period, and wave speed}
\begin{itemize}
    \item ($\lambda$) Wavelength: length of a complete oscillation.
    \item ($f$) Frequency: number of oscillations per second.
    \item ($T$) Period: time for one oscillation.
    \item ($v$) Wave speed: $v = \frac{\lambda}{T}$. The wave speed depends
        only on the properties of the medium and not on how it is produced.
\end{itemize}

\subsubsection{Transverse and longitudinal waves}
We call a wave transverse if the displacement is at right angles to the
direction of energy transfer.

In a longitudinal wave the displacement is parallel to the direction of energy
transfer.

A displacement-distance graph gives us the amplitude and wavelength of a wave.

A displacement-time graph gives us the amplitude and period of a wave.

A compression is a region of higher density; a rarefaction is a region of lower
density.

\subsubsection{The nature of electromagnetic waves}
All EM waves move at the speed of light in a vacuum.

\subsubsection{The nature of sound waves}
They are longitudinal.

% ================================
\subsection{Wave characteristics}

\subsubsection{Wavefronts and rays}
A wavefront is a surface through the crests and normal to the direction of
energy transfer of the wave. Lines in the direction of energy transfer of the
wave (and hence normal to the wavefronts) are called rays.

\subsubsection{Amplitude and intensity}
Intensity is the power incident on an area i.e $I = \frac{P}{a}$ Wm$^{-2}$.

For point sources of power the intensity at a distance $x$ becomes:
$I = \frac{P}{4\pi x^2}$.

We can see that $I \propto x^{-2}$, and so $I \propto A^2$.

\subsubsection{Superposition}
When two or more waves of the same type arrive at a given point in space at the
same time, the displacement of the medium at that point is the algebraic sum of
the individual displacements. So if $y_1$ and $y_2$ are individual
displacements, then at the point where the two meet the total displacement has
the value $y = y_1 + y_2$.

For pulses: fixed-end gives inverted pulse, free-end gives no change.

\subsubsection{Polarization}
An EM wave is said to be plane polarised if the electric field oscillates on
the same plane.

Unpolarsied light can be polarised by passing it through a polariser.

Malus's law: $E = E_0 \cos\theta$, or $I = I_0 \cos^2 \theta$.

Unpolarised light can also be partially polarised when it reflects off a
non-metallic surface.

% ================================
\subsection{Wave behaviour}

\subsubsection{Reflection and refraction}
The angle of incidence $i$ (angle between the ray and the normal to the
reflecting surface at the point of incidence) is equal to the angle of
reflection $r$ (angle between the normal and the reflected ray). The reflected
and incident rays and the normal to the surface lie on the same plane, called
the plane of incidence.

Refraction is the travel of light from one medium into another where it has a
different speed. 

\subsubsection{Snell's law, critical angle and total internal reflection}
Refraction changes the direction of the incident ray. Snell's law says:
$\frac{\sin\theta_2}{\sin\theta_1} = \frac{c_1}{c_2}$.

In the case of light, we define the refractive index of a given medium $n_m$
as: $n_m = \frac{c}{c_m}$.

For light, Snell's law can be rewritten:
$\frac{n_1}{n_2} = \frac{\sin\theta_2}{\sin\theta_1} = \frac{c_2}{c_1}$.

The angle of incidence for which the angle of refraction is 90$^{\circ}$ is
called the critical angle.

There is no refracted ray when the angle of incidence is greater than the
critical angle; there is just the reflected ray. This is known as total
internal reflection.

\subsubsection{Diffraction through a single-slit around objects}
Diffraction takes place when a wave with wavelength comparable to or larger
than the size of an aperture or an obstacle moves through or past the aperture
or obstacle. In general, the larger the wavelength, the greater the effect.

Example: Sound can diffract around doors because its wavelength is roughly the
same as the door size. Light cannot, since its wavelength is much smaller.

\subsubsection{Interference patterns}
Rays leaving different parts of the slit interfere and create complicated
intensity patterns, with a central maximum and then subsidiary maxima/minima
on either sides.

\subsubsection{Double-slit interference}
The resulting pattern (due to the principle of superposition) when two or more
waves meet is known as interference.

\textbf{Constructive interference} is when individual waves add up and the
amplitude is greater than for the individual waves (double if the waves are
identical). It occurs when the path difference is $n\lambda$, with
$n = 0, 1, 2, ...$. Waves are in phase.

\textbf{Destructive interference} is when indivdual waves cancel each other and
the amplitude is less than for the individual waves (0 if the waves are
identical). It occurs when the path difference is $(n + \frac{1}{2})\lambda$,
with $n = 0, 1, 2, ...$. Waves are exactly out of phase.

Double-slit interference results in a pattern of bright and dark spots on a
screen. The distance between two consecutive bright/dark spots is called
`fringe spacing', denoted $s$, given by: $s = \frac{\lambda D}{d}$ where $D$ is
the distance between slits and the screen, and $d$ is the distance between the
slits.

\subsubsection{Path difference}
Path difference is the difference in distance of a point from two sources. For
a point $P$ some distance away from two wave sources, $S_1$ and $S_2$, the
path difference will be $\delta r = |S_1P - S_2P |$.

% ================================
\subsection{Standing waves}

\subsubsection{The nature of standing waves}
When two waves of the same speed, wavelength, and amplitude travelling in the
opposite direction meet and interfere. The crests and troughs of a standing
wave stay in the same place.

Some observations about standing waves:
\begin{itemize}
    \item The wave does not `move' left or right, as travelling waves do.
    \item Points between consecutive nodes are in phase and have the same
        velocity. Points between the next pair of consecutive nodes have the
        opposite velocity.
    \item The amplitude of oscillation is different at difference points on
        the string.
    \item A standing wave does not transfer energy.
\end{itemize}

\subsubsection{Boundary Conditions}
The ends of a standing wave are either nodes or antinodes, which determine the
possible shape of the wave. These are known as boundary conditions. For
example, a string with both ends fixed has the conditions node-node, since
both ends are two nodes.

\subsubsection{Nodes and antinodes}
At some points of a standing wave, the displacement is always 0. These are
called nodes. The distance between two consecutive nodes is half a wavelength.

At some points, the displacement is as large as possible. These are called
antinodes.

A wave with node-node end conditions will have two nodes and one antinode. A
wave with this condition will have the longest wavelength (and lowest
frequency). It is called the first harmonic. The frequency of the first
harmonic is called the fundamental frequency.

All harmonics have frequencies that are integral multiples of the fundamental
frequency, i.e. of the first harmonic.

The wavelength for a standing wave with end conditions node-node or
antinode-antinode is given by $\lambda_n = \frac{2L}{n}$, $n = 1,2,3,4, ...$.

The wavelength for a standing wave with end conditions node-antinode or
antinode-node is given by $\lambda_n = \frac{4L}{n}$, $n = 1,3,5,7, ...$.

