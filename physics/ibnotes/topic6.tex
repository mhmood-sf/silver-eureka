% =============================================================================
\subsection{Circular Motion}

\subsubsection{Period, frequency, angular displacement and angular velocity}
\begin{definition}
    Period is the time taken to complete one full revolution, denoted by $T$.
    Since an object covers a distance of $2\pi r$ in time $T$ seconds, we have:
    \begin{align*}
        v = \frac{2 \pi r}{T}
    \end{align*}
    where $r$ is the radius of the circular path and $v$ is the linear speed.
\end{definition}

\begin{definition}
    The angular speed of an object, denoted $\omega$, is the angle swept per
    time taken:
    \begin{align*}
        \omega = \frac{\Delta \theta}{\Delta t}
    \end{align*}
    Its unit is radians per second, rads$^{-1}$.
\end{definition}

For one complete revolution, $\Delta \theta = 2\pi$ and $\Delta t = T$, so:
\begin{align*}
    \omega = \frac{2\pi}{T}
\end{align*}

Using the definition of frequency, $f = \frac{1}{T}$, we also have that
$\omega = 2\pi f$.

Using these equations we can also find the relationship between linear speed
and angular speed:
\begin{align*}
    v = r \omega
\end{align*}

\subsubsection{Centripetal force}
A body moving in a circle is always accelerating (since the direction of its
velocity is changing). Thus, there must also be a net force acting on it. For
constant speed, the direction of both the acceleration and force is towards
the centre of the circle. The magnitude of the force is given by:
\begin{align*}
    F = \frac{mv^2}{r}
\end{align*}

This is also equivalent to $F = m\omega^2 r$.

\textbf{Note:} centripetal force is not a new force. It is only a component of
the forces already acting on an object!

The work done by a centripetal force is 0, since the force is at right angles
to the direction of motion (recall that $W = Fd\cos\theta$).

\subsubsection{Centripetal acceleration}
\begin{definition}
    A body moving along a circle of radius $r$ with speed $v$ experiences
    centripetal acceleration that has magnitude given by:
    \begin{align*}
        a = \frac{v^2}{r}
    \end{align*}
    and is directed towards the centre of the circle.
\end{definition}

Some equivalent expressions for centripetal acceleration:
\begin{align*}
    a &= \omega^2 r \\
    a &= \frac{4 \pi^2 r}{T^2} \\
    a &= 4\pi^2 r f^2
\end{align*}

% =============================================================================
\subsection{Newton's law of gravitation}

\subsubsection{Newton's law of gravitation}
\begin{definition}
    The attractive force of gravitation between two point masses is given by:
    \begin{align*}
        F = G \frac{M_1 M_2}{r^2}
    \end{align*}
    where $M_1$ and $M_2$ are the masses of the attracting bodies, $r$ the
    distance between their centres of mass and $G$ a constant called Newton's
    constant of universal gravitation. It has the value
    $G = 6.667 \times 10^{-11}$ Nm$^2$kg$^{-2}$. The direction of the force is
    along the line joining the two masses.
\end{definition}

Equating the formula for centripetal force and the force of gravitation gives:
\begin{align*}
    v = \sqrt{\frac{GM}{r}}
\end{align*}
This is the speed of a particle mass orbiting a larger body of mass $M$ in a
circular orbit of radius $r$ (assuming the only force on the particle is the
force of gravitation).

\subsubsection{Gravitational field strength}
A mass $M$ is said to create a gravitational field in the space around it.
\begin{definition}
    The gravitational field strength at a certain point is the gravitational
    force per unit mass experienced by a small point mass $m$ placed at that
    point:
    \begin{align*}
        g = \frac{F}{m}
    \end{align*}
    The unit of gravitational field strength is Nkg$^{-1}$.
\end{definition}

We also have the gravitational field strength of a spherical mass $M$ given by:
\begin{align*}
    g = G \frac{M}{r^2}
\end{align*}

Gravitational field strength is also a vector quantity, whose direction is
towards the centre of the mass creating the field.

