% ================================
\subsection{Motion}

\subsubsection{Distance and displacement}
\begin{itemize}
    \item Displacement: change in position (vector)
    \item Distance: length of path followed (scalar)
\end{itemize}

\subsubsection{Speed and velocity}
\begin{itemize}
    \item Velocity: rate of change of displacement (vector)
    \item Speed: rate of change of distance (scalar)
\end{itemize}

\subsubsection{Acceleration}
Rate of change of velocity (vector).

\subsubsection{Graphs describing motion}
\begin{itemize}
    \item position/time graph: the slope gives the velocity; the area under
        the curve gives no useful quantity.
    \item velocity/time graph: the slope gives acceleration; the area under
        the curve gives displacement.
    \item acceleration/time graph: the slope gives no useful quantity (we're
        not interested in jerks); the area under the curve gives the velocity.
\end{itemize}

\subsubsection{Equations of motion for uniform acceleration}
\begin{itemize}
    \item Uniform motion: constant velocity.
    \item Uniform acceleration: constant acceleration.
\end{itemize}

For uniform acceleration \textit{only}, the following equations can be used:
\begin{align*}
           v &= u + at \\
    \Delta s &= ut + \frac{1}{2}at^2 \\
    \Delta s &= \left( \frac{u + v}{2} \right) t \\
         v^2 &= u^2 + 2a \Delta s
\end{align*}
Where $v$ is final velocity, $u$ is initial velocity, $s$ is position ($\Delta s$ is
displacement), $t$ is time and $a$ is acceleration.

\subsubsection{Projectile Motion}
A \textbf{projectile} is a body under the influence of only gravity, and no
other force.

Since only the downwards force of gravity is acting upon it, only
the vertical component of the acceleration (and velocity) is affected. There is
no acceleration along the horizontal axis, and the velocity along the
horizontal axis remains constant.

Use the equations of motion and vector
resolution to solve problems about projectile motion.

\subsubsection{Fluid Resistance and Terminal Speed}

Fluid resistance: Resistive force experienced when a body moves through a fluid
(gas or liquid). For small speeds, $F=kv$ and for high speeds, $F=kv^2$.

Terminal speed: As the fluid resistance force becomes larger, it eventually
equals the object's weight. The forces cancel out, acceleration becomes 0 and
the body moves at constant speed, which is known as the terminal speed. It can
be found using $mg = kv$.

% ================================
\subsection{Forces}

\subsubsection{Objects as point particles}
By treating a body as a point particle, we can assume that all forces act
through it at the same point.

\subsubsection{Free-body diagram}
To draw a free-body diagram for a chosen body, we treat it as a point particle
and draw arrows to represent all forces acting upon it. The arrows must be
proportional to the magnitude of the force. Surroundings and other bodies are
ignored.

\subsubsection{Translational Equilibrium}
We say an object is in translational equilibrium when all forces acting on it
add up to zero (i.e the net force is zero).

\subsubsection{Newton's Laws of Motion}

First law: When the net force on a body is zero, the body will move with
constant velocity. Also known as the law of inertia.

Second law: The net force on a body of constant mass is proportional to that
body's acceleration and is in the same direction as the acceleration.
Mathematically: $F=ma$.

Third law: If a body A exerts a force on body B, then body B will exert an
equal and opposite force on body A. Mathematically: $F_{ab} = -F_{ba}$.

\subsubsection{Solid Friction}
Friction is a force that opposes motion.
\begin{itemize}
    \item Dynamic/Kinetic friction: when there is motion e.g. one body slides
        over another. This is given by $f_d = \mu_d R$. R is the normal
        reaction force, $\mu_d$ is the coefficient of dynamic friction.
    \item Static friction: when there is a tendency for motion but not motion
        itself e.g. a body on a slope has a tendency to slide down, but is kept
        motionless by friction. The \textit{maximum} force of static friction
        is given by $f_s = \mu_s R$. R is the normal reaction force, $\mu_s$ is
        the coefficient of static friction.
\end{itemize}

Other important stuff:
\begin{itemize}
    \item $\mu_s > \mu_d$ (it takes more force to get an object to start
        sliding than to keep it sliding).
    \item Area of contact does not affect frictional force.
    \item Force of dynamic friction does not depend upon speed of sliding.
    \item Again, $f_s$ is the \textit{maximum} friction that can develop before
        an object will start sliding (once it does, the force acting will be
        dynamic friction).
\end{itemize}

% ================================
\subsection{Work, energy and power}

\subsubsection{Kinetic energy}
Energy of a particle due to its motion. $E_k = \frac{1}{2}mv^2$.

Work-kinetic energy relation: $W_{net} = \Delta E_k$.

\subsubsection{Gravitational Potential Energy}
Work done by the moving force in placing a body a height $h$ above its initial
position: $E_p = mgh$.

Gravitational potential energy is a conservative force, i.e the work done by
gravity is independent of the path followed and depends only on the vertical
distance separating the initial and final positions.

\subsubsection{Elastic Potential Energy}
Hooke's Law: the force $F$ and extension $x$ of a (stretched) spring are
directly proportional, i.e $F = kx$.

Work done by the pulling force in stretching a spring by an amount $x$:
$E_p = \frac{1}{2}kx^2$.

\subsubsection{Work done as energy transfer}
Work done by a force is the product of the force in the direction of the
displacement times the distance travelled, i.e $W = Fd\cos\theta$. Unit: joule
($1 J = 1 Nm$).

Work done by a force is the area under the graph that shows the variation of
the magnitude of the force with the distance travelled.

If a system is in contact with surroundings at a different temperature, there
will be a transfer of heat, $Q$. If there is no contact and no temperature
difference, then $Q = 0$. If no work is done from outside, then $W = 0$. When
$Q + W = 0$, the system is called \textit{isolated} and $\Delta E = 0$.

\subsubsection{Power as rate of energy transfer}
Power is the rate at which work is performed or the rate at which energy is
transferred: $P = \frac{\Delta W}{\Delta t}$. Unit: Watt (W),
$1 W = 1 Js^{-1}$.

Since $W = Fd$, we also have: $P = \frac{F \Delta d}{\Delta t} = Fv$.

\subsubsection{Principle of conservation of energy}
Energy is neither created nor destroyed. It is only transformed from one form
to another. The total amount of energy in the universe is constant.

\subsubsection{Efficiency}
Ratio of useful output energy/power to actual input of energy/power, i.e
$e = \frac{input}{output}$.

% ================================
\subsection{Momentum and Impulse}

\subsubsection{Newton's second law as rate of change of momentum}
Momentum (product of mass and velocity, vector): $p = mv$

Unit: kgms$^{-1}$ or Ns.

Newton's second law in terms of momentum: $F_{net} = \frac{\Delta p}{\Delta t}$
i.e average net force on a system is equal to the rate of change of the
momentum of the system.

\subsubsection{Impulse and force-time graphs}
Re-arranging Newton's second law in terms of momentum gives:
$\Delta p = F_{net} \Delta t$. This quantity is known as the impulse.

The larger the value of $\Delta t$, the smaller the net force will be, and vice
versa.

\subsubsection{Conservation of linear momentum}
When the net force on a system is zero the momentum does not change, i.e it
stays the same. We say it is conserved.

\subsubsection{Elastic collisions, inelastic collisions and explosions}
\begin{itemize}
    \item A collision between two objects is elastic when kinetic energy is
        conserved. It is perfectly elastic when all of the kinetic energy is
        conserved.
    \item A collision between two objects is inelastic when kinetic energy is
        not conserved. It is totally inelastic when all of the kinetic energy
        is lost.
\end{itemize}

Momentum also gives another useful formula for kinetic energy:
$E_k = \frac{p^2}{2m}$.

