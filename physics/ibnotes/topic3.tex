% ================================
\subsection{Thermal concepts}

\subsubsection{Molecular theory of solids, liquids and gases}
Solids, liquids and gases are made of smaller structures: compounds made of
molecules, molecules made of atoms and atoms containing nuclei and electrons.

In solids, particles are held together by bonds between them. In liquids these
bonds are weaker, and so the particles are able to slide and move around as
well as take the shape of the container. In gases the forces are very weak, and
so they are able to move around more freely allowing gases to expand as well.

\subsubsection{Temperature and absolute temperature}
Temperature is (proportional to) the average random kinetic energy of
molecules. It is measured in Kelvin. Absolute temperature is the lowest
possible temperature i.e 0 K. This is when the molecules have zero kinetic
energy.

Thermal equilibrium is when a chosen body and its surroundings (or another
body or bodies) reach the same temperature after energy is transferred between
them.

Heat is energy that is transferred from one body to another as a result of a
difference in temperature.

\subsubsection{Internal energy}
Internal energy is the total random kinetic energy of the particles of a
substance, plus the total inter-particle potential energy of the particles.

The internal energy of a system can change as a result of heat added or taken
out and as a result of work performed.

\subsubsection{Specific heat capacity}
The energy required to increase the temperature of a unit mass of the body by
one kelvin: $Q = mc\Delta T$.

\subsubsection{Phase change}
\begin{itemize}
    \item melting: solid to liquid
    \item freezing: liquid to solid
    \item vaporisation/boiling: liquid to gas
    \item condensation: gas to liquid
\end{itemize}
During the process of phase change, there is no change in the temperature of
the body (even as heat is being supplied to it).

\subsubsection{Specific latent heat}
Amount of energy required to change the phase of a unit mass at constant
temperature: $Q = mL$. For melting/freezing, it is called the specific latent
heat of fusion. For vaporisation/condensation it is called the specific latent
heat of vaporisation.

% ================================
\subsection{Modelling a gas}

\subsubsection{Pressure}
Normal force applied per unit area: $p = \frac{F}{A} = \frac{F\cos\theta}{A}$.

Unit: Pa (pascal) or atm (atmosphere, $1.013 \times 10^5$ Pa).

\subsubsection{Equation of state for an ideal gas}
Ideal gas laws:
\begin{itemize}
    \item Molecules are point particles with negligible volume.
    \item Molecules obey laws of mechanics.
    \item No forces between molecules except when they collide.
    \item Duration of collision is negligible compared to time between
        collisions.
    \item Collisions between molecules and molecules and the container walls
        are elastic.
    \item Molecules have a range of speeds and move randomly.
\end{itemize}
A real gas can be approximated by an ideal gas when the density is low.

The state of a gas is determined by the pressure (p), volume (V), temperature
(T), and number of moles (n). The relationship between them is given by the
equation of state: $pV = RnT$, where $R$ is the gas constant.

It is derived from the following:
\begin{itemize}
    \item Boyle's Law: $p \propto \frac{1}{V}$ or $pV = constant$.
    \item Charles' Law: $\frac{V}{T} = constant$.
    \item Amonton's Law: $\frac{p}{T} = constnat$.
\end{itemize}

\subsubsection{Kinetic model of an ideal gas}
The average random kinetic energy of particles is directly proportional to the
kelvin temperature: $\bar{E}_k = \frac{3}{2}\frac{R}{N_A}T$. The ratio
$\frac{R}{N_A}$ is also called the Boltzmann constant, $k_b$.

The total random kinetic energy (the internal energy) is given by:
$U = \frac{3}{2}Nk_bT$.

\subsubsection{Mole, molar mass and the Avogadro constant}
One mole of a substance contains as many particles as there are atoms in 12g of
carbon-12.

The number of particles in a mole is $N_A = 6.02 \times 10^{23}$ mol$^{-1}$.
This is known as the Avogadro constant.

The number of moles $n$ in a substance with $N$ particles is:
$n = \frac{N}{N_A}$.

The atomic mass unit (1 u) is: $\frac{1}{12}$ of the mass of one atom of
carbon-12. The mass of one atom of carbon-12 is exactly 12u. The atomic unit
mass is given (in grams) by: $u = \frac{1g}{N_A}$.

One mole of a substance is a quantity of the substance that contains a number
of particles equal to the Avogadro constant and whose mass in grams is equal to
the molar mass of the substance.

\subsubsection{Differences between real and ideal gases}
A real gas can be modelled as an ideal gas if it has low density.

