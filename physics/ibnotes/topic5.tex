% =============================================================================
\subsection{Electric Fields}

\subsubsection{Charge}
Charge is a property of matter. It can be negative (electrons) or positive
(protons).

Like charges repel and unlike charges attract. This force of attraction and
repulsion becomes weaker as the distance between the two bodies increases.

The SI Unit for charge is the Coulomb, symbol C.

Charge is quantised. The magnitude of the charge on the proton is equal to
$1.6 \times 10^{-19}$ C. This amount is symbolised by $e$. The charge of an
electron is $-e$.

Charge is conserved. It cannot be created or destroyed. The total charge in a
system cannot change.

Free electrons are those that do not belong to a particular atom and are able
to move about and carry charge through the metal. Materials with many free
electrons are called conductors. Those that do not are called insulators.

\subsubsection{Electric Field}
The space around a charge contains an electric field. This can be tested by
bringing a small test charge into the space and observing whether it
experiences a force or not.

Electric field strength is defined as the electric force per unit charge
experienced by a small, positive point charge:
\begin{align*}
    E = \frac{F}{q}
\end{align*}
Electric field is a vector quantity. The direction is the same as the force
experienced by the charge. The unit of electric field is N C$^{-1}$.

Using Coulomb's law (below) we can also find:
\begin{align*}
    E = k \frac{Q}{r^2}
\end{align*}

\subsubsection{Coulomb's Law}
The force between two charges is inversely proportional to the square of the
separation of the charges and proportional to the product of the two charges:
\begin{align*}
    F = k \frac{q_1q_2}{r^2}
\end{align*}

The constant $k$ is also written $\frac{1}{4\pi\epsilon_0}$, so that:
\begin{align*}
    F = \frac{1}{4\pi\epsilon_0} \frac{q_1q_2}{r^2}
\end{align*}
The factor $\frac{1}{4\pi\epsilon_0}$ is $8.99 \times 10^9$ N m$^2$ C$^{-2}$ in
a vacuum. The constant $\epsilon_0$ is the electric permittivity of vacuum and
$\epsilon_0 = 8.85 \times 10^{-12}$ C${^2}$ N$^{-1}$ m$^{-2}$. This constant
is different for different media.

\subsubsection{Electric Current}
The electric field inside a conductor is zero in static situations (when there
is no current). If an electric field is applied across it, the free electrons
experience a force which pushes them in the opposite direction to the field.

Electric current is defined as the rate of flow of charge through a conductor:
\begin{align*}
    I = \frac{\Delta q}{\Delta t}
\end{align*}
The unit of electric current is the ampere (symbol A), equal to flow of one
coulomb per second (1 A = 1 C s$^{-1}$).

For a conducting wire, the change in charge is given by $nAvq\Delta t$, where
$n$ is the number of electrons per unit volume, $A$ is the cross-sectional
area of the wire, $v$ is the drift speed, $q$ is the charge and $\Delta t$ is
the change in time.

Thus, for conducting wires:
\begin{align*}
    I = nAvq
\end{align*}
Electric current can be measured using an ammeter, connected to the device in
series. An ideal ammeter has zero resistance.

\subsubsection{Direct Current}
When electrons move in the same direction, it is known as direct current.

\subsubsection{Potential Difference}
The potential difference $V$ between two points is the work done per unit
charge to move a point charge from one point to another:
\begin{align*}
    V = \frac{W}{q}
\end{align*}
The unit of potential is the volt, V, and 1 V = 1 J C$^{-1}$.

Whenever there is a potential difference, there has to be an electric field.

The electronvolt is another unit for dealing with smaller charges:
\begin{align*}
    1 \text{eV} = 1.6 \times 10^{-19} \text{J}
\end{align*}
It is the work done in moving one electron charge across a potential difference
of one volt.

Potential difference can be measured using a voltmeter, connected in parallel
to the device. An ideal voltmeter has infinite resistance, so that no current
passes through it.

% =============================================================================
\subsection{Heating effects of electric currents}

\subsubsection{Circuit diagrams}
Refer to data booklet for symbols.

For resistors connected in series, the total resistance is:
\begin{align*}
    R_{total} = R_1 + R_2 + ... + R_n
\end{align*}

For resistors connected in parallel, the total resistance is:
\begin{align*}
    \frac{1}{R_{total}} = \frac{1}{R_1} + \frac{1}{R_2} + ... + \frac{1}{R_n}
\end{align*}

\subsubsection{Kirchhoff's circuit laws}
Kirchhoff's current law:
\begin{align*}
    \sum I_{in} = \sum I_{out}
\end{align*}
In words, the total current entering a junction is equal to the
total current exiting a junction.

Kirchhoff's loop law:
\begin{align*}
    \sum V = 0
\end{align*}
In words, the sum of voltages across each resistor is always equal to 0. This
is a consequence of the conservation of energy.

\subsubsection{Heating effect of current and its consequences}
An electric field within a conductor accelerates the free electrons. Electrons
gain kinetic energy and suffer inelastic collisions with the metal atoms. Energy
is transferred to the atoms and this results in an increase in the temperature
of the wire.

\subsubsection{Resistance expressed as $R = \frac{V}{I}$}
Electric resistance is a property of conductors which determines how much
current will flow for a given potential difference:
\begin{align*}
    R = \frac{V}{I}
\end{align*}
The unit is volt per ampere, also called the ohm, with symbol $\Omega$.

\subsubsection{Ohm's law}
When the temperature of most metallic conductors is kept constant, the current
through the conductor is proportional to the potential difference across it:
\begin{align*}
    I \propto V
\end{align*}
This is known as Ohm's law.

\subsubsection{Resistivity}
From experimentation, the electric resistance of a wire at a fixed temperature
has been found to be proportional to its length and inversely proportional to
its cross-sectional area:
\begin{align*}
    R = \rho \frac{L}{A}
\end{align*}
The constant $\rho$ is called resistivity, and depends on the conductor and the
temperature. The unit of resistivity is $\Omega$m.

\subsubsection{Power dissipation}
The potential difference at the ends of a resistor is commonly called
`voltage', and is given by $V = IR$.

Power is the rate of work done, and so the power dissipated in a resistor
is given by:
\begin{align*}
    P = IV = RI^2 = \frac{V^2}{R}
\end{align*}

% =============================================================================
\subsection{Electric Cells}

\subsubsection{Cells}
Cells or batteries are sources of potential difference that can be connected in
a circuit. They convert various forms of energy into electrical energy.

One characteristic of a cell is the amount of charge it can deliver in its
lifetime, known as its capacity. The bigger the current, the faster the cell
discharges. After an initial sudden drop, the terminal voltage remains almost
constant until the capacity is exhausted and there is again a sudden drop.

\subsubsection{Internal Resistance}
A real battery has internal resistance, denoted by $r$, connected to it in
series. It cannot be isolated from the battery. An ideal battery is one with
no internal resistance.

\subsubsection{Secondary Cells}
A primary cell is one which can only be used once, until it runs out. A
secondary cell is one which is rechargeable and can be used again.

\subsubsection{Terminal potential difference}
The current that leaves a real battery is $I$. The potential difference
across the internal resistance is $Ir$. Without the internal resistance, the
potential difference across the battery is its emf, $\epsilon$. Thus, the
voltage across the terminals of a real battery (including its internal
resistance) is:
\begin{align*}
    V = \epsilon - Ir
\end{align*}

\subsubsection{Electromotive force (emf)}
\begin{definition}
    The emf $\epsilon$ of a battery is the work done per unit charge in moving
    charge from one terminal of the battery to the other.
    \begin{align*}
        \epsilon = \frac{W}{q}
    \end{align*}
    It is also equal to power per unit current, i.e. $\epsilon = \frac{P}{I}$.
    The unit of emf is the volt, V.
\end{definition}

% =============================================================================
\subsection{Magnetic effects of electric currents}

\subsubsection{Magnetic fields}
A region of influence created by a magnet, similar to (but distinct from) an
electric field. A magnetic field can also be produced by an electric current.

The magnetic field, $B$, is a vector quantity.

We draw imaginary curves, called magnetic field lines, around a magnet to
depict the magnetic field. Magnetic field lines always exit from the north
pole, and enter at the south pole.

The direction of a magnetic field around a straight wire carrying a current
can be determined using the right-hand grip rule.

\subsubsection{Magnetic forces}
An electric charge moving through a region of magnetic field experiences a type
of force called a magnetic force.

There is no magnetic force on a moving charge if the charge moves along the
field direction (i.e. the velocity of the charge is parallel to the direction
of the field). In any other direction, there will be a force on the charge.

\begin{definition}
    If the magnetic force is $F$ on a charge $q$ moving with speed $v$, at an
    angle $\theta$ to the direction of the field, then the magnitude of the
    field, called magnetic flux density, is defined as:
    \begin{align*}
        B = \frac{F}{qv\sin\theta}
    \end{align*}
    The unit of magnetic flux density is the tesla (T).
\end{definition}

Thus, a charge $q$ moving with speed $v$ inside a magnetic field of magnetic
flux density $B$ will experience a magnetic force $F$ given by:
\begin{align*}
    F = qvB\sin\theta
\end{align*}

The right-hand rule can be used to determine the direction of the magnetic
field.

The magnetic force on a wire of length $L$ is given by:
\begin{align*}
    F = BIL\sin\theta
\end{align*}

Charges moving at right angles to a magnetic field follow the path of a
circle. No work is done by the magnetic force, since it is at a right angle to
the velocity of the charge (recall, $W = Fd\cos\theta$).



